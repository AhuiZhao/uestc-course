\documentclass[letterpaper]{article}
% \documentclass[runningheads]{llncs}
\usepackage{amsmath}
\usepackage{bm}
\usepackage{amssymb}
\usepackage{graphicx}
\usepackage{subfig}
\usepackage[UTF8, scheme=plain]{ctex}
\usepackage[colorlinks=true, linkcolor=red, filecolor=red, urlcolor=red, citecolor=cyan]{hyperref}
% \usepackage{amsfonts}
\usepackage{cleveref}
\usepackage{listings}
% \usepackage[htt]{hyphenat}
% \usepackage[newfloat]{minted}
% \usepackage{caption}
% \newenvironment{code}{\captionsetup{type=listing}}{}
% \SetupFloatingEnvironment{listing}{name=Snippet}
\usepackage{geometry}
\usepackage{multicol}
% \geometry{a4paper}
\setlength{\pdfpagewidth}{8.5in}  % DO NOT CHANGE THIS
\setlength{\pdfpageheight}{11in}  % DO NOT CHANGE THIS
\twocolumn
% code block
\geometry{a4paper,scale=0.95}
\setlength{\parindent}{0pt}
\frenchspacing
\graphicspath{{./imgs/}}
\allowdisplaybreaks
\newcommand{\theo}{\textbf{Theorem:}}
\begin{document}

\title{组合优化}
\author{Rongfan Li}
\date{}

\maketitle

\begin{abstract}
	Notes on combination optimal. 
\end{abstract}

\tableofcontents

\clearpage

\section{Linear programming}
% \input{1-lp.tex}
\subsection{单纯形法/对偶单纯形法}

1. 化成标准型。非规范形式可反号或者添加两个$x$解决
\begin{equation}\begin{aligned}
	&\max z=\sum c_{j} x_{j}\\
	&\text { s.t. }\left\{\begin{array}{c}
	\sum_{j=1}^{n} a_{i j} x_{j}=b_{i}(i=1,2, \cdots, m) \\
	x_{j} \geq 0 \quad(j=1,2, \cdots, n)
	\end{array}\right.
\end{aligned}\end{equation}


2. 化成对偶问题。注意非规范形式

3. 单纯形法/对偶单纯形法

4. 互补松弛性。在已知P的最优解的时候,可以用来求D的最优解

5. 人工变量法,两种思路

记号,原问题P,对偶问题D
\begin{equation}
	\begin{aligned}
		\operatorname{Max} Z=C X \quad & \text { Min } W=Y b\\
		\text { s.t. } \quad A X \leq b \quad & \text { s.t. } \quad Y A \geq C \\
		\quad X \geq 0 \quad & Y \geq 0
	\end{aligned}
\end{equation}

\theo 对偶的对偶就是原问题

\theo 若X和Y分别是可行解,则 $CX \leq YAX \leq Yb$,P的任意目标函数值是D的最优值的下界。若P有最优解,则D也有最优解,当等号取到,则分别为最优解,PD的最优解的值是相同的。

\theo 若P无有限的最优解(无界),则D无可行解。

除了上面两种情况,还有一种是两个都无可行解


\subsection{整数规划IP--决策变量是整数}

1. 割平面法,添加一个切割方程,p26,不断加约束直到b全为整数

2. 分支定界法。画图辅助理解问题。对原问题的最优解,按照变量12分两支进行整数约束,求出新的最优解,若是整数解则停止,若不是则递归


\section{Others}

\subsection{DP}

三个重要性质:最优子结构,重叠子问题,马氏性

\subsection{背包问题KP}

1. 分支定界法,标准化后的KP是一个整数规划问题

2. DP

\subsection{指派问题,匈牙利法}

对同一工作所有人的效率同时变化不影响最优分配。对同一人的所有工作效率同时变化也不影响最优分配。

\subsection{装箱问题}

最优解的下界:全部都恰好装入了箱子中。下面全是近似算法

1. NF/Next Fit。将$J_1$放入$B_1$后,如果$J_2$放不下,则直接封闭$B_1$并将$J_2$放入$B_2$。特点:封箱快,不需要全部物品装好后才能封箱搬运,适合场地小的安装。$O(n)$

2. FF/First Fit。对于$J_i$依次检查前面的全部$B$,放入第一个能装下$J_i$的箱子,若都放不下则启用空箱子。$O(n\log n)$

3. BF/Best Fit。$J_i$放入使得这个箱子剩余空间最小的那个箱子中。$O(n\log n)$

4. FFD/BFD,先按长度从大到小排序再按FF/BF处理$O(n\log n)$

FFD需要全部物品到了才能排序,不能用于在线装箱


\subsection{作业调度问题}

t是处理需要花费的时间,w是单位时间的损失,C是逗留时间,$C_{\max}$是时间表长/全体完工时间,d是限定的工期,L是延误时间

1. 单机调度

$1\| \sum w_j C_j$,总消耗最小最佳调度,$t/w$按照升序排列就是最佳调度。如果只考虑逗留时间,不考虑消耗,则视w=1

$1\| L_{\max}$,最大的延误时间最小,EDD 最早工期优先,按d升序。

$1\| \sum U$,延误的任务最少,按照EDD先排序,将延误任务和前面的全部任务中t最大的任务放最后。

2. 平行机调度 PMS-m个机器并行

$Pm\|C_{\max}$,最早完工时间,是NP-hard的,只有近似算法

LS算法,将每一个J分给最早空闲的机器。适合在线调度

LPT算法,按照加工时间t从大到小排列,再用LS(因而无法在线)

3. 车间作业调度,多类型机

$Fm\|C_{\max}$,m个机器流水线作业/每个作业有m道工序,共n个作业

$F2\|C_{\max}$,Johnson/SPT-LPT算法,先将全部作业按照工序1花费时间是否比2少来分成两个集合,少的在J1中,按照工序1不减排列,J2集合中按照工序2不增排列。见eg.12

\subsection{最大流问题}

s源点,t汇点,c容量,f流量,截集:两个不相容的点集之间的弧

增广路径:从s到t的无向弧集(前向和后向),并且都不饱和

$\theo$ 最大流的流值和最小截的容量相同

$\theo$ f是最大流 $\iff$ D中没有f的增广路径


% \bibliographystyle{plain}
% \bibliography{./bib/doc.bib}

\end{document}
